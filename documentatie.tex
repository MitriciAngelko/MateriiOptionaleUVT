\documentclass[12pt,a4paper]{article}
\usepackage[utf8]{inputenc}
\usepackage[romanian]{babel}
\usepackage{hyperref}
\usepackage{listings}
\usepackage{xcolor}
\usepackage{graphicx}
\usepackage{float}

% Configurare pentru cod
\lstset{
    basicstyle=\ttfamily\small,
    breaklines=true,
    numbers=left,
    numberstyle=\tiny,
    frame=single,
    keywordstyle=\color{blue},
    commentstyle=\color{green!60!black},
    stringstyle=\color{red},
    showstringspaces=false
}

\title{Documentație Tehnică\\Sistem de Management Universitar}
\author{Universitatea de Vest din Timișoara}
\date{\today}

\begin{document}

\maketitle
\tableofcontents
\newpage

\section{Introducere}
Acest document descrie implementarea tehnică a sistemului de management universitar dezvoltat pentru Universitatea de Vest din Timișoara. Sistemul este construit folosind tehnologii moderne web și oferă funcționalități pentru gestionarea materiilor, utilizatorilor și relațiilor dintre aceștia.

\section{Arhitectura Sistemului}
\subsection{Tehnologii Utilizate}
\begin{itemize}
    \item Frontend: React.js cu Redux pentru management-ul stării
    \item Backend: Firebase (Firestore, Authentication)
    \item Stilizare: Tailwind CSS
\end{itemize}

\section{Structura Bazei de Date}
\subsection{Colecții Firestore}
\begin{itemize}
    \item \textbf{users}
    \begin{itemize}
        \item ID: generat automat
        \item email: string
        \item nume: string
        \item prenume: string
        \item tip: string (student/profesor/admin)
        \item materiiPredate: array (pentru profesori)
    \end{itemize}
    
    \item \textbf{materii}
    \begin{itemize}
        \item ID: generat automat
        \item nume: string
        \item facultate: string
        \item specializare: string
        \item an: string
        \item credite: number
        \item descriere: string
        \item profesori: array
    \end{itemize}
\end{itemize}

\section{Componente Principale}
\subsection{Autentificare și Autorizare}
Sistemul implementează următoarele tipuri de utilizatori:
\begin{itemize}
    \item Admin (@admin.com)
    \item Profesori (@e-uvt.ro)
    \item Studenți (alte domenii)
\end{itemize}

\subsection{Navbar}
Componenta de navigare oferă acces contextual la funcționalități bazat pe rolul utilizatorului:
\begin{itemize}
    \item Toți utilizatorii: Home, Profil
    \item Profesori: Materiile Mele
    \item Admin: Administrare Utilizatori, Administrare Materii
\end{itemize}

\subsection{Pagini Principale}
\subsubsection{MateriileMelePage}
Permite profesorilor să:
\begin{itemize}
    \item Vizualizeze materiile la care sunt asignați
    \item Acceseze detalii complete despre fiecare materie
    \item Vadă lista de studenți înscriși (funcționalitate viitoare)
\end{itemize}

\subsubsection{AdminMateriiPage}
Permite administratorilor să:
\begin{itemize}
    \item Adauge materii noi
    \item Editeze materii existente
    \item Șteargă materii
    \item Filtreze materiile după facultate, specializare și an
\end{itemize}

\section{Funcționalități de Securitate}
\begin{itemize}
    \item Autentificare bazată pe Firebase Authentication
    \item Rutare protejată pentru pagini specifice rolurilor
    \item Validare pe partea de client și server
\end{itemize}

\section{Interfața Utilizator}
\subsection{Design System}
\begin{itemize}
    \item Culori principale: 
    \begin{itemize}
        \item Primary: \#034a76 (Albastru UVT)
        \item Secondary: \#023557
    \end{itemize}
    \item Sistem de grid responsiv
    \item Componente reutilizabile stilizate cu Tailwind CSS
\end{itemize}

\section{Dezvoltări Viitoare}
\begin{itemize}
    \item Implementarea sistemului de înscriere la materii pentru studenți
    \item Adăugarea unui sistem de notificare
    \item Implementarea unui sistem de încărcare documente
    \item Sistem de evaluare și feedback
\end{itemize}

\section{Concluzii}
Sistemul oferă o bază solidă pentru managementul academic, cu posibilități extinse de dezvoltare și scalare. Arhitectura modulară și tehnologiile moderne utilizate permit adăugarea facilă de noi funcționalități și mentenanță eficientă.

\end{document} 